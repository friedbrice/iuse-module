\documentclass[10pt,aspectratio=1610,xcolor={dvipsnames}]{beamer}

\usepackage[utf8]{inputenc}
\usepackage[T1]{fontenc}
\usepackage{standalone}
\usepackage{fbpicture}
\usepackage{microtype}
\usepackage{amsmath, amssymb}

\usefonttheme[onlymath]{serif}

\title{Panda Population Growth}

\begin{document}

  \begin{frame}
    \titlepage
  \end{frame}

  \begin{frame}{Meet the Logistics Curve Family}

    \begin{center}
      \scalebox{0.75}{% DESCRIPTION:
% KEYWORDS:
\documentclass[border=0.5in]{standalone}

\usepackage{fbpicture}

\begin{document}
    \linethickness{0.5pt}
  \begin{fbpic}{-2}{8}{-1}{4}
    % axes
    \put(0,-1){\vector(0,1){5}}
    \put(-2,0){\vector(1,0){10}}
    % dots
    \multiput(-0.1,3)(0.4,0){21}{\line(1,0){0.2}}
    \multiput(-0.1,1.5)(0.4,0){8}{\line(1,0){0.2}}
    % points
    \put(3.1,1.5){\circle*{0.1}}
    % labels
    \put(8.1,-0.1){$t$}
    \put(-0.2,4.1){$f(t)$}
    \put(-0.4,2.9){$A$}
    \put(-0.8,1.4){$A/2$}
    % graph
    \linethickness{1pt}
    \qbezier(3.1,1.5)(1.9,0.1)(-2,0.1)
    \qbezier(3.1,1.5)(4.3,2.9)(8,2.9)
  \end{fbpic}
\end{document}
}
    \end{center}

    \begin{columns}

      \column{0.5\textwidth}
      \begin{center}
        A \emph{logistics curve} is a function
        \[
          f(t) = \frac{A}{1 + Be^{-Ct}}
        \]
        where $A$, $B$, and $C$ are constants.
      \end{center}

      \column{0.5\textwidth}
      The important properties are:

      \begin{itemize}
        \item{$f(t) \to A$ as $t \to \infty$}
        \item{$f(t) \to 0$ as $t \to -\infty$}
        \item{concave up when $f(t) < A / 2$}
        \item{concave down when $f(t) > A / 2$}
        \item{max growth rate when $f(t) = A / 2$}
      \end{itemize}

    \end{columns}

  \end{frame}

  \begin{frame}{Logistics Curves Model Population Growth}

    \begin{columns}

      \column{0.5\textwidth}
      \begin{center}
        \scalebox{0.65}{% DESCRIPTION:
% KEYWORDS:
\documentclass[border=0.5in]{standalone}

\usepackage{fbpicture}

\begin{document}
    \linethickness{0.5pt}
  \begin{fbpic}{-2}{8}{-1}{4}
    % axes
    \put(0,-1){\vector(0,1){5}}
    \put(-2,0){\vector(1,0){10}}
    % dots
    \multiput(-0.1,3)(0.4,0){21}{\line(1,0){0.2}}
    \multiput(-0.1,1.5)(0.4,0){8}{\line(1,0){0.2}}
    % points
    \put(3.1,1.5){\circle*{0.1}}
    % labels
    \put(8.1,-0.1){$t$}
    \put(-0.2,4.1){$f(t)$}
    \put(-0.4,2.9){$A$}
    \put(-0.8,1.4){$A/2$}
    % graph
    \linethickness{1pt}
    \qbezier(3.1,1.5)(1.9,0.1)(-2,0.1)
    \qbezier(3.1,1.5)(4.3,2.9)(8,2.9)
  \end{fbpic}
\end{document}
}
      \end{center}

      \column{0.5\textwidth}
      \begin{center}
        Population growth is an important topic of study in the subfield of Biology called \emph{Ecology}.
        \[
          f(t) = \frac{A}{1 + Be^{-Ct}}
        \]
        Logistics curves are useful for modeling population growth.
      \end{center}

    \end{columns}

    \begin{itemize}
      \item{$f(t)$ represents the number of individuals living at time $t$.}
      \item{$A$ represents the \emph{carrying capacity} (the maximum number of individuals that the environment can support).}
      \item{When $f(t) < A / 2$, resources are plentiful, and the population growth rate increases.}
      \item{When $f(t) > A / 2$, resources are scarce, and competition for resources drives the population growth rate down.}
    \end{itemize}

  \end{frame}

  \begin{frame}{Example: The Mountain Lions in Yosemite}

    \begin{columns}

      \column{0.5\textwidth}
      A scientist is modeling the population of mountain lions in Yosemite National Park using a logistics curve with $t$ measured in years and $t=0$ corresponding to 2000.
      \[
        f(t) = \frac{A}{1 + Be^{-Ct}}
      \]
      Through observation and data analysis, she has determined that the constants $A = 4000$, $B = 10$, and $C = 0.12$ are a good fit for her observed data.

      \column{0.5\textwidth}
      Use your calculator to plot the model from $t = 0$ to $t = 50$.

      Now take a few minutes to answer the following:

      \begin{enumerate}
        \item{What is the carrying capacity for this population?} % 4000
        \item{At what population level will the growth rate be the highest?} % 2000
        \item{At what time will the growth rate be the highest?} % t = 2019.188
        \item{What is the maximum growth rate (and what unit is that in)?} % 120 ml / yr
      \end{enumerate}

    \end{columns}

  \end{frame}

  \begin{frame}{Solution: The Mountain Lions in Yosemite}

    \begin{columns}

      \column{0.5\textwidth}
      \begin{enumerate}
        \item{
          $A$ represents the carrying capacity, so the carrying capacity is 4000.
        }
        \item{
          The logistics curve has it's fastest rate of growth when $f(t)$ is $A / 2$, so the population will have the highest growth rate when there are 2000 mountain lions.
        }
        \item{
          We need to find the value of $t$ when $f(t) = 2000$.
          \begin{align*}
            2000 &= \frac{4000}{1 + 10e^{-0.12t}} \\
            1 + 10e^{-0.12t} &= 2 \\
            e^{-0.12t} &= 0.1 \\
            -0.12t &= \ln(0.1) \\
            t &= \frac{\ln(0.1)}{-0.12} \\
            t &\approx 19.188
          \end{align*}
        }
      \end{enumerate}

      \column{0.5\textwidth}
      \begin{enumerate}
        \setcounter{enumi}{3}
        \item{
          Since $f(t)$ tells us the population, the derivative $f'(t)$ will tell us the growth rate of the population.
          \begin{align*}
            f(t) &= \frac{4000}{1 + 10e^{-0.12t}} \\
            f'(t) &= \frac{d}{dt} \frac{4000}{1 + 10e^{-0.12t}} \\
            f'(t) &= \frac{0 - 4000(-1.2e^{-0.12t})}{(1 + 10e^{-0.12t})^2} \\
            f'(t) &= \frac{4800e^{-0.12t}}{(1 + 10e^{-0.12t})^2} \\
            f'(19.188...) &= 120
          \end{align*}
          Since $f(t)$ is mountain lions in terms of time (in years), $f'(t)$ is mountain lions per year.
        }
      \end{enumerate}

    \end{columns}

  \end{frame}

\end{document}
