\documentclass{letter}

\usepackage[key]{fbtest}
\usepackage{amsmath}
\usepackage{graphicx}

\begin{document}

    We are in charge of repairing the population of pandas in a nature
    reserve in inland China.
    The panda carrying capacity in our reserve is 165, but there
    are currently only 5 pandas in the reserve.

    The population is modeled by the logistics curve
    \[
      P(t) = \frac{165}{1 + 32 e^{-0.2t}}
    \]
    where $t$ is measured in years. (Take a moment to verify that the
    current population $P(0)$ is indeed 5.)

    We intend to reintroduce zoo-bred pandas to our reserve in order
    to increase the population; however, if we
    increase the panda population too quickly, then we risk causing an
    ecological disaster that will destroy the reserve's ecosystem.

    \begin{problem}{}
      Find the rate of growth of the panda population.
      About how many pandas can we expect to be born this year
      (i.e. the year $t = 0$)?

      \solution{
        \begin{align*}
          P(t) &= \frac{165}{1 + 32e^{-0.2t}} \\
          P'(t) &= \frac{1056 e^{0.2 t}}{(e^{0.2 t} + 32)^2} \\
          P'(0) &= 0.969...
        \end{align*}

        We can expect about one panda to be born this year.
      }
    \end{problem}

    \begin{problem}{}
      Find the time $t$ at which the population has the fastest
      rate of growth.
      (Remember that a logistics curve has its maximum rate of growth
      when the population is at half the carrying capacity.)

      \solution{
        Since the carrying capacity is 165,
        we need to solve $P(t) = 82.5$.

        \begin{gather*}
          \frac{165}{1 + 32 e^{-0.2t}} = 82.5 \\
          t = 25 \ln(2) \\
          t \approx 17.328
        \end{gather*}
      }
    \end{problem}

    \begin{problem}{}
      Find the maximum rate of growth of the population.

      \solution{
        We found that the max rate of growth occurs at $t = 25\ln(2)$.

        \begin{align*}

        \end{align*}
      }
    \end{problem}

    We can assume that the natural maximum rate of growth is a
    ecologically safe growth rate.

    \begin{problem}{}
      How many pandas can we safely introduce to the reserve this year,
      assuming that the zoo-bred pandas do not breed the first year?
      (Remember, you need to account for the naturally-born offspring
      of the 5 initial pandas.)
    \end{problem}

    \begin{problem}{}
      After we introduce as many pandas as we can, what can we expect
      the population to be one year from now? What will the rate of
      growth be at that time?
    \end{problem}

\end{document}
