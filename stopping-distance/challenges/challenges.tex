\documentclass{letter}

\usepackage[key]{fbtest}
\usepackage{amsmath}
\usepackage{graphicx}

\begin{document}

  \begin{problem}{}
    Our car is travelling at a constant velocity of 30 m/s when we spot
    a priceless painting on the road 100 m in front of us! We apply the
    brakes, giving our car an acceleration of $-$5 m/s\textsuperscript{2}.

    Putting our initial position as 0 m and the position of the painting
    as 100 m, find our position when the car comes to a stop (i.e., when
    velocity is 0). Did we save the painting?

    \solution{
      \begin{align*}
        a(t) &= -5 \\
        v(t) &= -5t + v_0 \\
        &= -5t + 30 \\
        x(t) &= -2.5t^2 + 30t + x_0 \\
        &= -2.5t^2 + 30t
      \end{align*}

      The car comes to a stop when $v(t) = 0$.

      \begin{gather*}
        0 = v(t) \\
        0 = -5t + 30 \\
        t = 6
      \end{gather*}

      We need to know if we hit the painting (which is at position 100).
      We calculate the car's position when it stops.

      \begin{gather*}
        x(t) = -2.5t^2 + 30t \\
        x(6) = 90
      \end{gather*}

      We stopped at position 90, so we saved the painting!
    }
  \end{problem}

  \begin{problem}{}
    We are designing a car, and we need to ensure that it has a stopping
    distance of 50 m when traveling at a speed of 25 m/s. How much
    acceleration must the brakes be designed to provide?

    \solution{
      \begin{align*}
        a(t) &= -A \\
        v(t) &= -At + v_0 \\
        &= -At + 25 \\
        x(t) &= -\frac{A}{2} t^2 + 25t + x_0 \\
        &= -\frac{A}{2} t^2 + 25t
      \end{align*}

      We need to find the time $t$ when the velocity is zero.

      \begin{gather*}
        0 = v(t) \\
        0 = -At + 25 \\
        t = \frac{25}{A}
      \end{gather*}

      Our car stops at time $t = \frac{25}{A}$. We need to see how far
      it goes.

      \begin{align*}
        x(t) &= -\frac{A}{2} t^2 + 25t \\
        x \left( \frac{25}{A} \right)
        &= -\frac{A}{2} \left( \frac{25}{A} \right)^2 + 25\left( \frac{25}{A} \right) \\
        &= \frac{625}{2A}
      \end{align*}

      So our stopping distance is $\frac{652}{2A}$. On the other hand,
      we need our stopping distance to be 50 m. Equate, and solve for $A$.

      \begin{gather*}
        50 = \frac{625}{A} \\
        A = 6.25
      \end{gather*}

      We need our brakes to be able to apply $-6.25$ m/s\textsuperscript{2}
      of acceleration.
    }
  \end{problem}

  \begin{problem}{}
    Our car has brakes that can provide $-$6 m/s\textsuperscript{2}
    acceleration. Find the stopping distance as a function of velocity,
    and find the velocity at which the stopping distance becomes 100 m
    (give your answer in m/s and correct to 2 decimal places).
  \end{problem}

\end{document}
