\documentclass{letter}

\usepackage[key]{fbtest}
\usepackage{amsmath}
\usepackage{graphicx}

\begin{document}

  \begin{problem}{}
    Our car is travelling at a constant velocity of 30 m/s when we spot
    a priceless painting on the road 100 m in front of us! We apply the
    brakes, giving our car an acceleration of $-$5 m/s^2.

    Putting our initial position as 0 m and the position of the painting
    as 100 m, find our position when the car comes to a stop (i.e., when
    velocity is 0). Did we save the painting?
  \end{problem}

  \begin{problem}{}
    We are designing a car, and we need to ensure that it has a stopping
    distance of 50 m when traveling at a speed of 25 m/s. How much
    acceleration must the brakes be designed to provide?
  \end{problem}

  \begin{problem}{}
    Our car has brakes that can provide $-$6 m/s^2 acceleration. Find
    the stopping distance as a function of velocity, and find the
    velocity at which the stopping distance becomes 100 m (give your
    answer in m/s and correct to 2 decimal places).
  \end{problem}

\end{document}
